\chapter{Convertidor elevador}

Una vez definidos los convertidores reductores, una topología similar a estos se puede apreciar en los choppers DC elevadores que permiten elevar el nivel DC de salida empleando para ello un inductor en la etapa de entrada y que se utiliza para almacenar la energía en forma de campo magnético mediante las variaciones de corriente debido a la inductancia que esta genera por su composición.

\section{Convertidor elevador con carga R}

Para estudiar el comportamiento de este tipo de convertidor es necesario detallar su estructura, viéndose la misma en la figura \ref{fig:convertidor-elevador} en la cual se muestra una bobina o inductor de entrada un switch, un diodo rectificador y finalmente la carga.

\begin{figure}[h]
	\centering
	\includegraphics[width=0.5\linewidth]{media/convertidor-elevador}
	\caption{Convertidor Elevador - Chopper DC}
	\label{fig:convertidor-elevador}
\end{figure}

\subsection{Principio de funcionamiento}

El funcionamiento de este elevador se ejecuta por etapas o pasos considerando como elemento principal para elevar el voltaje la bobina capaz de almacenar energía, elevando así el voltaje de salida; la secuencia de pasos a considerar se detalla a continuación:

\begin{enumerate}
	\item Cuando el switch se encuentra en funcionamiento el inductor se carga
	\item Cuando el interruptor se apaga la energía almacenada es liberada incrementando el nivel de voltaje DC en la salida 
\end{enumerate}

\subsection{Análisis del circuito}

Este circuito al contar con 2 etapas de funcionamiento en función al switch o alternador se contarán con 2 circuitos diferentes según la conmutación, considerando para el caso 1 cuando la compuerta de cortocircuito un circuito en serie con la fuente de alimentación de entrada y el inductor, mientras que en el caso contrario que el inductor se encuentra en serie con la carga, esto a grandes rasgos se muestra en la figuras \ref{fig:step-up-mode1} y \ref{fig:step-up-mode2} respectivamente.

\begin{figure}[h]
	\centering
	\includegraphics[width=0.2\linewidth]{media/step-up-mode1}
	\caption{Modo 1 - Convertidor elevador}
	\label{fig:step-up-mode1}
\end{figure}

\begin{figure}[h]
	\centering
	\includegraphics[width=0.2\linewidth]{media/step-up-mode2}
	\caption{Modo 2 - Convertidor elevador}
	\label{fig:step-up-mode2}
\end{figure}













