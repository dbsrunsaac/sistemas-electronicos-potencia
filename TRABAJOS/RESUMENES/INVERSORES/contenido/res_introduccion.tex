\chapter{Introducción}

Los inversores monofásicos de medio puente constituyen una topología fundamental en la electrónica de potencia debido a su simplicidad estructural y a su valor didáctico y práctico en la conversión de energía DC–AC. Estos convertidores son ampliamente utilizados en aplicaciones de baja y media potencia, tales como sistemas de alimentación ininterrumpida (UPS), accionamientos de motores monofásicos, fuentes conmutadas y etapas inversoras en sistemas fotovoltaicos aislados, donde se requiere una solución eficiente y de bajo costo. Su importancia radica en la reducción del número de dispositivos semiconductores respecto a topologías de puente completo, lo que simplifica el control y disminuye las pérdidas por conmutación. No obstante, presentan limitaciones relevantes, entre ellas la necesidad de un bus DC partido con punto medio capacitivo, el riesgo de desbalance de tensión en los condensadores y una menor utilización del voltaje del bus DC en comparación con inversores de puente completo, aspectos que condicionan su desempeño y restringen su uso en aplicaciones de mayor potencia, tal como se documenta ampliamente en la literatura científica especializada en electrónica de potencia.
