\chapter{Conversor Step Down-buck}
\section{Principio de Funcionamiento}
El regulador buck utiliza un interruptor, transistor MOSFET, un diodo, un inductor y un condensador para transferir energía de manera controlada desde la entrada a la salida. El interruptor se enciende y apaga a una frecuencia determinada, permitiendo que el inductor almacene y libere energía, lo que resulta en un voltaje de salida regulado y más bajo que el de entrada.
\subsection{Interruptor encendido.}
Cuando el interruptor se activa, el diodo queda polarizado en inversa debido al voltaje de entrada. En consecuencia, toda la corriente de entrada fluye a través del inductor. Así, la corriente continua de entrada (Idc) que recorre el circuito es igual a la corriente que pasa por el inductor:

\ IL = Idc

Durante el tiempo en que el interruptor está encendido, el inductor almacena energía. La corriente que circula por él se reparte entre la corriente que va hacia la carga (Io) y la corriente que carga o descarga el condensador (Ic).
\begin{figure}[h]
	\centering
	\includegraphics[width=0.5\linewidth]{media/stepBuck_1.png}
	\caption{Caption}
	\label{fig:placeholder}
\end{figure}

\subsection{Interruptor Apagado.}
Cuando el interruptor se apaga, el circuito pasa del modo 1 al modo 2. 
En este modo la polaridad del inductor se invierte y comienza a entregar 
la energía almacenada, por lo que actúa como una fuente. La corriente 
fluye gracias a dicha energía, ya que la fuente de entrada queda 
desconectada. La corriente continúa circulando hasta que el inductor 
se descarga por completo. Durante este intervalo, el voltaje en el 
inductor es igual al voltaje de salida, pero con signo negativo:

\[
V_{L_{\text{off}}} = -V_o
\]

Al apagarse el interruptor, la inversión de polaridad del inductor provoca 
la polarización directa del diodo, haciendo que el ánodo quede más 
positivo que el cátodo y permitiendo su conducción.
\begin{figure}[h]
	\centering
	\includegraphics[width=0.5\linewidth]{media/stepBuck_2.png}
	\caption{Caption}
	\label{fig:placeholder}
\end{figure}
\section*{Clasificación de corriente del interruptor en un convertidor Buck}

La corriente nominal del interruptor se determina a partir del valor promedio 
de la corriente que circula por él. Para obtener este valor, se analiza la 
forma de onda de la corriente del interruptor y se calcula su promedio.

La corriente del inductor puede expresarse como la suma de la corriente del 
interruptor y la corriente del diodo, de acuerdo con la Ley de Corrientes de 
Kirchhoff (LCK). Durante el intervalo de encendido, la corriente del inductor 
coincide con la corriente del interruptor, mientras que durante el intervalo de 
apagado coincide con la corriente del diodo.

Las formas de onda de la corriente del interruptor \(I_{\text{switch}}\), 
la corriente del diodo \(I_{\text{diode}}\) y la corriente del inductor \(I_L\) 
se muestran para los tiempos de encendido y apagado.

La corriente promedio del interruptor está dada por:

\[
\langle I_{\text{switch}} \rangle 
= \frac{1}{T}\int_{0}^{DT} I_L(t)\, dt
\]

Sustituyendo los valores correspondientes, se obtiene:

\[
\langle I_{\text{switch}} \rangle 
= \langle I_L \rangle \cdot D
\]

Dado que la corriente promedio del inductor es igual a la corriente promedio 
de salida:

\[
\langle I_L \rangle = \langle I_o \rangle
\]

la expresión queda como:

\[
\langle I_{\text{switch}} \rangle 
= \langle I_o \rangle \cdot D
\]

Finalmente, la ecuación de clasificación de corriente máxima para el 
interruptor es:

\[
I_{\text{switch,max}} = I_{o,\text{max}} \cdot D
\]




– Explica su principio de operación, ciclo de trabajo y comportamiento de la corriente, debe contener el desarrollo y deduccion de formulas.

– Describe su funcionamiento, ecuaciones básicas y relación de conversión, debe contener el desarrollo y deduccion de formulas.
