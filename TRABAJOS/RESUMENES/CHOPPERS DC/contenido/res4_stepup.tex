\chapter{Convertidor elevador}

Una vez definidos los convertidores reductores, una topología similar a estos se puede apreciar en los choppers DC elevadores justificado en \cite{rashid_power_electronics} que permiten elevar el nivel DC de salida empleando para ello un inductor en la etapa de entrada y que se utiliza para almacenar la energía en forma de campo magnético mediante las variaciones de corriente debido a la inductancia que esta genera por su composición.

\section{Convertidor elevador con carga R}

Para estudiar el comportamiento de este tipo de convertidor es necesario detallar su estructura, viéndose la misma en la figura \ref{fig:convertidor-elevador} en la cual se muestra una bobina o inductor de entrada un switch, un diodo rectificador y finalmente la carga.

\begin{figure}[h]
	\centering
	\includegraphics[width=0.5\linewidth]{media/convertidor-elevador}
	\caption{Convertidor Elevador - Chopper DC}
	\label{fig:convertidor-elevador}
\end{figure}

\subsection{Principio de funcionamiento}

El funcionamiento de este elevador se ejecuta por etapas o pasos considerando como elemento principal para elevar el voltaje la bobina capaz de almacenar energía, elevando así el voltaje de salida; la secuencia de pasos a considerar se detalla a continuación:

\begin{enumerate}
	\item Cuando el switch se encuentra en funcionamiento el inductor se carga
	\item Cuando el interruptor se apaga la energía almacenada es liberada incrementando el nivel de voltaje DC en la salida 
\end{enumerate}

\subsection{Análisis del circuito}

Este circuito al contar con 2 etapas de funcionamiento en función al switch o alternador se contarán con 2 circuitos diferentes según la conmutación, considerando para el caso 1 cuando la compuerta de cortocircuito un circuito en serie con la fuente de alimentación de entrada y el inductor, mientras que en el caso contrario que el inductor se encuentra en serie con la carga, esto a grandes rasgos se muestra en la figuras \ref{fig:step-up-mode1} y \ref{fig:step-up-mode2} respectivamente.

\begin{figure}[h]
	\centering
	\includegraphics[width=0.2\linewidth]{media/step-up-mode1}
	\caption{Modo 1 - Convertidor elevador}
	\label{fig:step-up-mode1}
\end{figure}

\begin{figure}[h]
	\centering
	\includegraphics[width=0.2\linewidth]{media/step-up-mode2}
	\caption{Modo 2 - Convertidor elevador}
	\label{fig:step-up-mode2}
\end{figure}

Cabe destacar que en cada caso el interruptor en paralelo a la etapa de salida define cada modo, siendo que para el primer caso genera un corto circuito hacia tierra separando así la carga de la entrada permitiendo almacenar la energía.

Por lo tanto para cada modo es necesario realizar el análisis eléctrico y posteriormente el matemático considerando además los fenómenos físicos que se experimentan en los componentes que lo integran.

Al aplicar la L.T.K en la malla se relaciona el voltaje del inductor con el voltaje de entrada obteniendo así una expresión que relaciona la variación en la la corriente respecto al voltaje y el tiempo, recordando así que la inductancia es la oposición frente al flujo de corriente variable.
 
\begin{align}
	-V_S + V_L = 0 \\
	L\frac{di}{dt} = V_S \\
	L\frac{\Delta I}{\Delta T} = V_S \\
	\Delta I = \frac{V_S*t_1}{L} \\
	\label{eq:mode1-deltaI}
\end{align}

En el modo 2 cuando el interruptor esta abierto se tiene un circuito en serie polarizando en directa el diodo $D_m$ hacía la carga descargándose en tal instante de tiempo $t_2$ el inductor elevando el voltaje de salida; Al aplicar L.T.K en circuito de la figura \ref{fig:step-up-mode2} se obtienen la relación \textit{considerando un voltaje umbral equivalente a cero en el diodo} que vincula el voltaje de entrada con los diferentes elementos del circuito.

\begin{align}
	V_O - V_L = V_S\\
	V_S + V_L = V_O\\
	V_S + L\frac{\Delta I}{\Delta T} = V_O \\
	\Delta T = t2
\end{align}

En función a la ecuación \ref{eq:mode1-deltaI} se obtiene un valor el voltaje de salida considerando la variación en la corriente debido a la carga y descarga del inductor, relacionando así los diferentes voltajes.

\begin{align*}
	V_O &= V_S + L*\frac{V_S*t_1}{Lt_2}\\
	V_O &= V_S + V_S*t_1	\\
	V_O &= V_S*(1 + \frac{t_1}{t_2})\\
	t_1 &= KT\\
	t_2 &= (1-K)T\\
	V_O &= V_S(1 + \frac{K}{1-K})\\
\end{align*}

\begin{equation}
	V_O = \frac{V_S}{1-K}
	\label{eq:vo-salida-stepup}
\end{equation}

La expresión final para el voltaje de salida definida en \ref{eq:vo-salida-stepup}  nos indica una dependencia directa con el ciclo de trabajo del conmutador, por lo tanto a mayor \% en el ciclo de trabajo se tendrá un mayor voltaje de salida.

\section{Convertidor elevador con carga R y fuente E}

En comparación al caso anteriormente definido con carga R, en la topología la diferencia se encontrará en la salida agregando una fuente de voltaje DC (E), lo cual agregará más detalles al análisis matemático y físico puesto que ahora se debe considerar un estado inicial en el circuito el cual se desconoce y el cual se puede deducir a partir de los teoremas del valor inicial y final para su análisis.

Un modelo topologico de este tipo se aprecia en la figura \ref{fig:step-up-circuito-re} siendo sus modelos equivalentes los mostrados previamente, sin embargo se puede destacar para este caso el uso de un BJT para la etapa de conmutación.

\begin{figure}[h]
	\centering
	\includegraphics[width=0.3\linewidth]{media/step-up-circuito-RE}
	\caption{Conversor elevador - Carga RE}
	\label{fig:step-up-circuito-re}
\end{figure}

Para el análisis de corriente el añadir una fuente de voltaje no modifica los fenomenos físicos o compartiendo asociado a este parámetro, por lo tanto las curvas de transferencia no se ven alteradas considerando 2 rectas, una de pendiente positiva durante la carga y otra de pendiente negativa durante la descarga lo cual evidencia de forma gráfica un \textit{cambio en la polaridad en la bobina durante la descarga} implicando al mismo tiempo un cambio en la dirección de la corriente, esto se ve reflejado en la figura \ref{fig:curvas-corriente-inductor}

\begin{figure}[h]
	\centering
	\includegraphics[width=0.4\linewidth]{media/curvas-corriente-inductor}
	\caption{Curvas de corriente en el inductor - carga y descarga}
	\label{fig:curvas-corriente-inductor}
\end{figure}

Al considerarse nuevamente un circuito conmutado es necesario evaluar 2 casos en el circuito, estando delimitado cada etapa por el ciclo del trabajo de conmutación cada una equivalente a $t_1$ y $t_2$, siendo así que en entre $ 0 < t \leq t1 $ se cuenta con el circuito de la figura \ref{fig:step-up-mode1}, aplicando la L.T.K se relacional los voltajes de entrada y el voltaje del inductor, considerando esta vez la ecuación diferencial correspondiente para el inductor.

\begin{align*}
	V_S &= V_L \\
	V_L &= L \frac{di_1(t)}{dt} \\
	di_1(t) &= \frac{V_S}{L}dt \\
	\int{di_1(t)} &= \int{\frac{V_S}{L}dt}\\
\end{align*}

\begin{equation}
	i_1(t) = \frac{V_S}{L}t + A
	\label{eq:i1-carga-stepup}
\end{equation}

Para obtener la constante A considerada en la ecuación \ref{eq:i1-carga-stepup} y limitar la familia de funciones resultantes de la solución de la ecuación diferencial se tendrá en cuenta el teorema del valor inicial y para ello se tendrá que para un tiempo $t = 0$ la corriente $i_1(t) = I_1$.

\begin{align}
	I_1 = A \\
	i_1(t) = \frac{V_S}{L}t + I_1
\end{align}

Obteniendo de esta forma el valor de $i_1(t)$ en función del voltaje de entrada e $I_1$, luego considerando que el valor del corriente varia durante la descarga para obtener el valor pico máximo $I_2$ se hará uso del teorema del valor final definiendo $t = KT$ e $i_1(t) = I_2$ de tal relación se tiene:

\begin{equation}
	I_2 = \frac{V_S}{L}KT + I_1
	\label{eq:i2-carga-stepup}
\end{equation}

En la ecuación \ref{eq:i2-carga-stepup} se define el pico máximo para la corriente $I_2$, sin embargo para obtener la curva de corriente correspondiente a la descargar del inductor es imperativo hacer uso del circuito para el modelo 2 mostrado en \ref{fig:step-up-mode2} cuando el parámetro de tiempo varía entre $ KT < t \leq (1-K)T$ equivaliendo la diferencia entre ambos extremos a un tiempo $t_2$.

Aplicando la L.T.K en el circuito serie resultante se obtienen la siguiente relación de voltajes considerando para ello la fuente DC (E).

\begin{align*}
	V_L + i_2R + E = V_S \\
	L\frac{di_2(t)}{dt} + i_2R + E = V_S\\
\end{align*}

De la ecuación diferencial resultante se implica que su solución contempla una función divida en 2 etapas la transitoria y en régimen permanente otorgando así la división de la corriente $i_2(t)$ mediante el esquema

\begin{equation}
	i_2(t) = i_s + i_{tr}
	\label{eq:i2-respuesta-stepup}
\end{equation}

La solución de la Ec. diferencial en función a la expresión de control definida en \ref{eq:i2-respuesta-stepup} nos permite evidenciar que su parte constante es el valor final en régimen estacionario y la exponencial la parte transitoria puesto que cuanto más tiempo transcurra este valor ira constantemente en decremento hasta anularse \textit{idealmente}.

\begin{equation}
	i_2(t) = \frac{V_S - E}{R} + Ae^{-\frac{t}{\tau}}
	\label{eq:i2-resp-control}
\end{equation}

La ecuación \ref{eq:i2-resp-control} define la solución de la Ec. diferencial siendo en esta el valor de $\tau = \frac{L}{R}$ el cual detalla la dependecia del tiempo de estabilidad en función al valor de la inductancia y resistencia, sin embargo al igual que en el primer caso como solución se tiene una familia de función, por lo que se volverá aplicar el teorema del valor inicial y final.

Aplicando el teorema del valor inicial se espera que para un tiempo $t = 0$ la corriente $i_2(t) = I_2$, en tal sentido al evaluar \ref{eq:i2-resp-control} bajo estas condiciones, se tiene que:

\begin{align*}
	I_2 &= \frac{V_S - E}{R} + A \\
	A &= I_2 -  \frac{V_S - E}{R}\\
	Finalmente \\
	i_2(t) &= I_2e^{-\frac{R}{L}t} + \frac{V_S - E}{R}(1 - e^{-\frac{R}{L}t}
\end{align*}

Al obtener el valor de $i_2(t)$ se puede continuar con la siguiente etapa obteniendo para ello el valor $I_1$ en la etapa de descarga siendo así que el teorema del valor final podrá proporcionar esta valor al considerar $t2 = (1-K)T$ e $i_2(t) = I_1$

\begin{equation}
	I_1 = I_2e^{-\frac{R}{L}(1-K)T} + \frac{V_S - E}{R}(1 - e^{-\frac{R}{L}(1-K)T})
	\label{eq:i1-down-current}
\end{equation}

En la ecuación \ref{eq:i1-down-current} se describe la ecuación exponencial que describe la descargar del circuito elevador \textit{relacionado al inductor} y en función al ciclo de trabajo y el periodo de la señal.

Finalmente a partir de las ecuaciones \ref{eq:i2-carga-stepup} y \ref{eq:i1-down-current} se puede obtener una expresión general para la corriente rizo $\Delta I$ \ref{eq:deltai-stepup} ($ \Delta I = I_2 - I_1 $)

\begin{equation}
	 \Delta I = \frac{V_S}{L}KT 
	\label{eq:deltai-stepup}
\end{equation}










